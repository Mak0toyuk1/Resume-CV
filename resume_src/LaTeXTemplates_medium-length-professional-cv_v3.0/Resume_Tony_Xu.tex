%%%%%%%%%%%%%%%%%%%%%%%%%%%%%%%%%%%%%%%%%
% Medium Length Professional CV
% LaTeX Template
% Version 3.0 (December 17, 2022)
%
% This template originates from:
% https://www.LaTeXTemplates.com
%
% Author:
% Vel (vel@latextemplates.com)
%
% Original author:
% Trey Hunner (http://www.treyhunner.com/)
%
% License:
% CC BY-NC-SA 4.0 (https://creativecommons.org/licenses/by-nc-sa/4.0/)
%
%%%%%%%%%%%%%%%%%%%%%%%%%%%%%%%%%%%%%%%%%

%----------------------------------------------------------------------------------------
%	PACKAGES AND OTHER DOCUMENT CONFIGURATIONS
%----------------------------------------------------------------------------------------

\documentclass[
	%a4paper, % Uncomment for A4 paper size (default is US letter)
	11pt, % Default font size, can use 10pt, 11pt or 12pt
]{resume} % Use the resume class

%\usepackage{ebgaramond} % Use the EB Garamond font
\usepackage{xcolor}
\usepackage{hyperref}
% \usepackage{setspace}
% \onehalfspacing % or \doublespacing


%------------------------------------------------

\name{Tony Xu} % Your name to appear at the top

% You can use the \address command up to 3 times for 3 different addresses or pieces of contact information
% Any new lines (\\) you use in the \address commands will be converted to symbols, so each address will appear as a single line.

%\textsf{\address{766 king st.w\\ Hamilton, ON} % Main address

\address{Hamilton, ON} % A secondary address (optional)}

\address{(365)~$\cdot$~366~$\cdot$~5128 \\ xua32@mcmaster.ca} % Contact information

\address{\href{http://www.linkedin.com/in/tony-xu-87751529a}{Linkedin Profile}}

% \address{\href{https://github.com/Mak0toyuk1}{Github}}





%----------------------------------------------------------------------------------------

\begin{document}




%----------------------------------------------------------------------------------------
%	EDUCATION SECTION
%----------------------------------------------------------------------------------------

\begin{rSection}{Education}
	
	\textbf{Honours Bachelour of science, Mathematics and Statistics Co-op} \hfill \textit{2021 - present} \\ 
	Mcmaster university \\
	Relevant coursework: \begin{itemize}
		\item Mathematical Scientific computation, 
		\item Probability
		\item Ordinary differential equation
		\item Statistical Inference
		\item Introduction to modelling
		\item Applied linear regression with SAS
		\item Introduction to machine learning and multivariate analysis
	\end{itemize}
	%Minor in Linguistics \smallskip \\
	%Member of Eta Kappa Nu \\
	%Member of Upsilon Pi Epsilon \\
	%Overall GPA: 8.2
	
\end{rSection}


%----------------------------------------------------------------------------------------
%	WORK EXPERIENCE SECTION
%----------------------------------------------------------------------------------------

\begin{rSection}{Academic Projects}

	\begin{rSubsection}{Python}{September 2022}{}{}
		\item Implemented the random search, coordinate search, coordinate descent and gradient descent algorithms using the numpy and matplotlib packages.
	\end{rSubsection}

	% \begin{rSubsection}{C++}{May 2019}{}{}
	% 	\item Programmed a calculator in C++.
	% 	\item Rock paper and scissors game.
	% 	\item Computed the fibonacci sequence.
	% \end{rSubsection}

%------------------------------------------------

	% \begin{rSubsection}{MATLAB}{September 2022}{}{}
	% 	\item Applied the simplex method to a set of variables.
	% 	\item Computed matrix products.
	% 	\item Computed eigenvalues.
	% 	\item Implemented the Gram-Schmidt method.
	% 	\item Computed the QR factorization of a matrix.
	% 	\item Computed the orthogonal projections of a matrix.
	% 	\item Computed the Singular value decomposition.
	% \end{rSubsection}

	\begin{rSubsection}{R}{September 2024}{}{}
		\item Used dimensionality reduction techniques such as PCA and FA in exploratory data analysis.
		\item Used Supervised learning methods such as k-nearest neighbours, classfication trees and random forests to explain data
		\item Used Logistics regression to obtain the odds ratio to interpret data.

	\end{rSubsection}
	
	% \begin{rSubsection}{\LaTeX}{Present}{}{}
	% 	\item Used \LaTeX for every homework that requires a mathematical proof from second year to present. ($30$+ homeworks) 
	% 	\item This resume is compiled using \LaTeX .
	% \end{rSubsection}

	
	

%------------------------------------------------

	

\end{rSection}

\begin{rSection}{Volunteer Experiences}
	\begin{rSubsection}{Research Assistant}{July 2024 - August 2024}{}{}
		\item Third author in Computational Measures of Gaze Behavior Using the Concept of Situational Awareness with instructor: Qing Xu.
		\item Read various documentations in the field of partial information decomposition to give feedback and corrections.
		\item Helped typeset the \LaTeX document.
	\end{rSubsection}
	
\end{rSection}

% \newpage

%----------------------------------------------------------------------------------------
%	Extracurriculars
%----------------------------------------------------------------------------------------

% \begin{rSection}{Extracurriculars / Volunteer Experiences}
% 	% \begin{rSubsection}{High school badminton team, Windsor ON}{September 2018- June 2019}{}{Windsor, ON}
% 	% 	\item Assisted coach with setting up the field every training day.
% 	% 	\item Trained 2 hours each day for 2 days a week.
% 	% 	\item Maintained a positive relation with coach and team members by giving feedback of practice matches.
		
% 	% \end{rSubsection}
	
% 	\begin{rSubsection}{High school band, Windsor ON}{September 2018 - June 2021}{}{Windsor, ON}
% 		\item Became proficient with the clarinet and guitar over the course of 3 years.
% 		\item Performed in several smaller projects within the band.
% 		\item Performed at the Windsor spitfire games with the entire band.
% 		\item Maintained a good relation with the band conductor(high school music teacher) by seeking advices on performing and giving constructive criticism on the entire band.
		
% 	\end{rSubsection}

	
% \end{rSection}

%----------------------------------------------------------------------------------------
%	Skills section
%----------------------------------------------------------------------------------------

\begin{rSection}{Skills}

	\item \textbf{Technical skills}: Microsoft Office, SAS, MATLAB, Git
	\item \textbf{Programming software}: Python, Elementary C++, R, and \LaTeX
	\item \textbf{Languages}: Fluent in English and Mandarin.
\end{rSection}

%----------------------------------------------------------------------------------------

%----------------------------------------------------------------------------------------
%	EXAMPLE SECTION
%----------------------------------------------------------------------------------------

%\begin{rSection}{Section Name}

	%Section content\ldots

%\end{rSection}

%----------------------------------------------------------------------------------------

\end{document}
